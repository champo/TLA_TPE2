\documentclass[11pt,a4paper,titlepage]{article}
\usepackage[utf8]{inputenc}
\usepackage[spanish]{babel}
\usepackage{amsmath}
\usepackage{amsfonts}
\usepackage{amssymb}
\usepackage{makeidx}
\usepackage{graphicx}
\usepackage{tipa}

\title{Conversor de Gramáticas - Autómatas Finitos}
\author{Civile, Juan \and Sneidermanis, Dario \and Alvaro, Crespo}
\date{24 de Junio del 2012}

\begin{document}

\newcommand{\awesome}[1]{\texttt{\large #1}}

\maketitle
\tableofcontents
\clearpage

\section{Consideraciones}
Se hizo una pequeña modificación a la sintaxis de los archivos de gramáticas, para poder implementarla (los nombres de gramáticas no admiten caracteres `\texttt{=}`).

\section{Desarrollo}
Comenzamos por adaptar el archivo \texttt{.lex} utilizado en el primer trabajo para las gramaticas.
Esto requirio adaptar las estructuras de datos que guardan la gramatica procesada, ya que asumian una gramatica regular.

Luego comenzamos por implmentar una version muy sencilla del generador.
Esta primer version solo soportaba gramaticas \texttt{LL(0)}.
Con eso hecho, procedimos a dar soporte para todo tipo de gramatica libre de contexto.

Finalmente, modificamos el algoritmo para que sea capaz de imprimir la lista de producciones utilizadas para analizar la sentencia.

\section{Dificultades}
Pasar del pseudocodigo generico del analizador a una implementacion funciona no fue trivial.
La implementacion dada en las diapositivas no realiza el retroceso correctamente.
De hecho, esa implementacion solo soporta gramaticas \texttt{LL(0)}.
Por ejemplo, para una gramatica con producciones \texttt{A -> aBd}, \texttt{B -> b | bc}, considera que \texttt{abcd} no es parte del lenguaje generado.

% Agregar algo mas sobre como lo solucionamos

\section{Extensiones}
Una posible extension es generar analizadores predictivos.
Se podria intentar detectar si el lenguaje es \texttt{LL(k)} para algun \texttt{k} entre 0 y un numero razonable.
Y con ese conocimiento, generar un analizador predictivo para ese tipo de gramatica.
En caso de que la gramatica no caiga bajo ninguna de esas categorias, se volveria al uso de retroceso.

\end{document}
