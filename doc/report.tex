\documentclass[11pt,a4paper,titlepage]{article}
\usepackage[utf8]{inputenc}
\usepackage[spanish]{babel}
\usepackage{amsmath}
\usepackage{amsfonts}
\usepackage{amssymb}
\usepackage{makeidx}
\usepackage{graphicx}
\usepackage{tipa}

\title{Conversor de Gramáticas - Autómatas Finitos}
\author{Civile, Juan \and Sneidermanis, Dario \and Alvaro, Crespo}
\date{24 de Junio del 2012}

\begin{document}

\newcommand{\awesome}[1]{\texttt{\large #1}}

\maketitle
\tableofcontents
\clearpage

\section{Consideraciones}
Se hizo una pequeña modificación a la sintaxis de los archivos de gramáticas, para poder implementarla (los nombres de gramáticas no admiten caracteres `\texttt{=}`).

\section{Desarrollo}
Comenzamos por adaptar el archivo \texttt{.lex} utilizado en el primer trabajo para las gramaticas.
Esto requirio adaptar las estructuras de datos que guardan la gramatica procesada, ya que asumian una gramatica regular.

Something, something. Something.

\section{Dificultades}

\section{Extensiones}
Implementar un predicto maybe?

\end{document}
