\documentclass[11pt,a4paper,titlepage]{article}
\usepackage[utf8]{inputenc}
\usepackage[spanish]{babel}
\usepackage{amsmath}
\usepackage{amsfonts}
\usepackage{amssymb}
\usepackage{makeidx}
\usepackage{graphicx}
\usepackage{tipa}

\title{Generador de Analizador Sintáctico Descendente con Retroceso}
\author{Civile, Juan \and Sneidermanis, Dario \and Alvaro, Crespo}
\date{24 de Junio del 2012}

\begin{document}

\newcommand{\awesome}[1]{\texttt{\large #1}}

\maketitle
\tableofcontents
\clearpage

\section{Consideraciones}
Se hizo una pequeña modificación a la sintaxis de los archivos de gramáticas, para poder implementarla (los nombres de gramáticas no admiten caracteres `\texttt{=}`).

Si bien el enunciado no presentaba ninguna restricción al respecto, decidimos utilizar ISO C99 para el código generado.

\section{Desarrollo}
Comenzamos por adaptar el archivo \texttt{.lex} utilizado en el primer trabajo para las gramáticas.
Esto requirió adaptar las estructuras de datos que guardan la gramática procesada, ya que asumían una gramática regular.

Luego comenzamos por implmentar una versión muy sencilla del generador.
Esta primer versión solo soportaba gramáticas \texttt{LL(0)}.
Con eso hecho, procedimos a dar soporte para todo tipo de gramática libre de contexto.

Finalmente, modificamos el algoritmo para que sea capaz de imprimir la lista de producciones utilizadas para analizar la sentencia.

\section{Dificultades}
Pasar del pseudocodigo genérico del analizador a una implementación funcional no fue trivial.
La implementación dada en las diapositivas no realiza el retroceso correctamente.
De hecho, esa implementación solo soporta gramáticas \texttt{LL(0)}.
Por ejemplo, para una gramática con producciones \texttt{A -> aBd}, \texttt{B -> b | bc}, considera que \texttt{abcd} no es parte del lenguaje generado.

% Agregar algo mas sobre como lo solucionamos

\section{Extensiones}
Una posible extension es generar analizadores predictivos.
Se podría intentar detectar si el lenguaje es \texttt{LL(k)} para algún \texttt{k} entre 0 y un número razonable.
Y con ese conocimiento, generar un analizador predictivo para ese tipo de gramática.
En caso de que la gramática no caiga bajo ninguna de esas categorías, se volvería al uso de retroceso.

\end{document}
